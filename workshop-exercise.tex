\documentclass{article}

%Packages

\usepackage[utf8]{inputenc}

\usepackage[parfill]{parskip}

\usepackage[colorlinks=true,urlcolor=codeblue]{hyperref}

\usepackage{minted, graphicx}

%Document

\title{`Software Sustainability: If a tree falls in a forest...' -
King's workshop exercise} 

\author{Martin Chapman}

\date{Wednesday 9th December 2020}

\begin{document}

\definecolor{bg}{rgb}{0.95,0.95,0.95}
\definecolor{codeblue}{rgb}{0,0.3,0.6}

\maketitle

\section{Prerequisites}

The below material assumes that the software sustainability course has been completed (\href{https://www.youtube.com/playlist?list=PLxyHJ\_wep1\_DPbvtFl\_-EGyoz2pVt-n1\_}{https://www.youtube.com/playlist?list=PLxyHJ\_wep1\_DPbvtFl\_-EGyoz2pVt-n1\_}), and that Python 3, nodejs, Git, Docker and Docker Compose are all installed. 

\section{Background}

\begin{center}
    \includegraphics[height=5cm]{weather.png}
\end{center}

We are going to imagine we have just invented a perfect model of
the weather, that allows us to make predictions for a specific
location on a specific day.

Our model takes as input a location (e.g. ``London'') and the
number of days from now at which to make a prediction about
the weather in that location (e.g. 3 (days from now)).

To make a prediction, the model first takes the \href{https://en.wikipedia.org/wiki/List_of_Unicode_characters}{Unicode} value of
the first letter of the location (e.g. `L', which is 76 in
Unicode) and computers the remainder of this value when divided
by 3. This produces a number between 0 and 2. Next, the model
takes the number of days from now at which to make a prediction,
and also computes the remainder of this value when divided by 3.
This also produces a number between 0 and 2. These two numbers
are then combined to produce a number between 0 and 4.

To make a final prediction, each number produced by the model is matched to
a specific type of weather:

0 = sun, 1 = wind, 2 = rain, 3 = snow and 4 = cloud.

\subsection{Example}

\begin{enumerate}
    \item A user enters \mintinline[bgcolor=bg]{bash}{London} and \mintinline[bgcolor=bg]{bash}{3}.
    \item The value of `L' in Unicode is 76. The remainder of 76 / 3 is 1.
    \item The remainder of 3 / 3 is 0.
    \item Our final prediction value is 1 (1 + 0), which corresponds to `wind'.
\end{enumerate}

\section{Session 1 (10.10 - 10.30) - Coding practice}

Realise the weather model described in Python code, adhering to the principles
of coding practice discussed in the course.

Hint: the \mintinline[bgcolor=bg]{python}{ord()} function is
useful for converting a character to its Unicode value.

Your code should accept a location and a number of days as user input,
and use the model to make a prediction based on this input.

\section{Session 2 (11.00 - 11.25) - Version control}

Make a version of your software, and push it to a remote repository
on King's Github (\href{https://github.kcl.ac.uk/}{github.kcl.ac.uk}).
Call this repository `\textbf{simulation}'.

You will need to log in with your k-number and password, if you
have not done so before.

\textit{Public Github (\href{https://github.com}{github.com}), should also work for this task,
but should only be used as a backup.}

Additional versions should be made at appropriate points while completing the remaining tasks.

\subsection{Assessment}

To receive a mark for your work,
add the user \textbf{MC} as a collaborator to your repository.

To add someone as a collaborator, from your repository 
visit \textit{Settings} $>$ \textit{Collaborators \& Teams}. You may
need to log in again. In the \textit{Collaborators} box, under
\textit{Search by username, full name or email address}, search for the
user `\textbf{MC}'. Click the first result in the dropdown box and then
click \textit{Add collaborator}. Verify that `MC' is a collaborator on
your repository.

\textit{If using public Github as a backup, the path to add a collaborator will instead be \textit{Settings} $>$
\textit{Manage access}. Click \textit{Invite a collaborator}, and 
in the box that pops up search for the user `\textbf{martinchapman}' and click the top result. Verify that `martinchapman' is a collaborator on your repository.}

If you are asked to set an access level for the collaborator, choose \textit{Admin}.

Via this collaborator link, your repository will be accessed and 
your mark will be based on the number of passing tests
in your code.

\section{Session 3 (12.00 - 12.25) - Testing}

Write three tests to ensure the following:

\begin{enumerate}
    \item The weather in London in 5 days time is ``snow''.
    \item The weather in London in 365 days time is ``snow''.
    \item The weather in London in -1 days time is not reported.
\end{enumerate}

Modify your program such that these tests pass, if needed.

You are welcome to add additional tests, but these must all pass.

\section{Session 4 (14.00 - 14.25) - Services}

The supporting material for this session is available at \href{https://github.com/martinteaching/sustainability2020}{github.com/martinteaching/sustainability2020}.

Wrap your Python weather prediction model in a server and run it.

Move the request for user input (location and days) to a Javascript client,
which, once acquired, sends this information to the server, waits for it to compute
the result using the model, and then prints the result for the user.

\section{Session 5 (15.00 - 15.25) - Docker}

Dockerise your program, using a Dockerfile and Docker compose, allowing the Python server to run in a container, and the Javascript client to issue requests to it.

\section{Additional Tasks}

If you wish to expand on your solution, separate your Python weather model server code into
two servers (services). Each service should contain code to compute part of 
the weather prediction (e.g. one service for the location calculation, another
for the day calculation). Designate one of your services as the `user-facing'
service, to be called by the client (Service A). Service A, when called by a client, should then subsequently call the other service (Service B) for its result, before combining Service B's result with its own, and returning this in a response to the client. 

For example, Service A
could compute the location value, call Service B to compute the day value, combine
the results, and return them to a requesting client.

\end{document}

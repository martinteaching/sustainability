\documentclass[10pt, dvipsnames, table, aspectratio=169]{beamer}

\usetheme[progressbar=frametitle]{metropolis}

\usepackage{appendixnumberbeamer}

\usepackage{booktabs}

\usepackage[scale=2]{ccicons}

\usepackage{pgfplots}

\usepgfplotslibrary{dateplot}

%Project imports:

\usepackage[color=yellow]{todonotes}

\usepackage{hyperref}

\urlstyle{same}  % don't use monospace font for urls

\usepackage[]{minted}

%Project code:

\hypersetup{unicode=true,pdftitle={},pdfauthor={},pdfkeywords={},pdfborder={0 0 0},breaklinks=true}

\setbeamertemplate{frame footer}{
    \hfill\href{https://bit.ly/sustain25}{bit.ly/sustain25}
}

%Document:

\title{Software Sustainability}

\subtitle{Workshop}

\date{May 2nd}

\author{Martin Chapman}

\institute{King's College London}

\begin{document}

\definecolor{bg}{rgb}{0.95,0.95,0.95}

\maketitle

%==============================

{
\setbeamertemplate{footline}{}
\begin{frame}[fragile]{Format}

    The \href{https://github.com/martinteaching/sustainability#material}{\textbf{online material}}, which you have hopefully already reviewed, is the `\textbf{lecture}'.
    
    \begin{itemize}
    
        \item Available on \textbf{YouTube} (recorded offline and via live stream).
    
        \item If you have not yet had an opportunity to review this material, there will be \textbf{time today} to do so.
    
    \end{itemize}
    
    Today is the `\textbf{tutorial}'.
    
    \begin{itemize}
    
        \item A \href{https://github.com/martinteaching/sustainability/tree/master/workshops/kcl/2025}{\textbf{workshop}} in which we further explore the ideas from the course.
    
    \end{itemize}
    
    Links to all material available at:
    \href{https://bit.ly/sustain25}{\textbf{bit.ly/sustain25}}.
    
\end{frame}
}

%==============================

{
\setbeamertemplate{footline}{}
\begin{frame}[fragile]{Summary: Software sustainability}

\textbf{Definition}: The ability of software to \textbf{endure}.

\textbf{Useful for us because}: We want our research software to have \textbf{continued impact}.

\textbf{Achieved by}:

\begin{center}
    \begin{tabular}{p{4.2cm}|p{2.7cm}|p{5.8cm}}

      \textbf{Property} & \textbf{Realisation} & \textbf{Example}  \\

      \hline

      Maintainability & Coding practice & Comments = \textbf{understandable} code. \\

      \hline

      Extensibility & Version control & Git provides a \textbf{path of updates}. \\

      \hline

      Trust & Testing & Unit tests can show that our code works in a \textbf{wide number of situations}.  \\

      \hline

      Usability and Interoperability & Services & HTTP requests can be made \textbf{easily} and from \textbf{any language}. \\

      \hline

      Portability and Scalability & Containerisation & Docker containers can be \textbf{run anywhere} and can be \textbf{replicated}. \\

      \hline

\end{tabular}

\end{center}

\textbf{Note}: Sustainable (i.e. environmentally friendly) software is important too!

\end{frame}
}

%==============================

\begin{frame}[fragile]{Workshop structure}

A series of \textbf{five 50-minute sessions}.
Each session will relate to one of the five topics covered in the course:

\begin{centering}

\begin{table}[h!]

    \centering

    
    
    \begin{tabular}{c|c|c}

         \textbf{Time} & \textbf{Activity} & \textbf{Platform}  \\

         \hline

         10.00 - 10.10 & Introduction & YouTube \\

         \hline

         10.10 - 10.30 & Coding practice: exercise & Teams \\

         \hline

         10.30 - 10.50 & Live coding solution & YouTube \\

         \hline

         10.50 - 11.00 & Break & - \\

         \hline

         11.00 - 11.25 & Version control: exercise & Teams \\

         \hline

         11.25 - 11.50 & Live coding solution & YouTube \\

         \hline

         11.50 - 12.00 & Break & - \\

         \hline

         12.00 - 12.25 & Testing: exercise & Teams \\

         \hline

         12.25 - 12.50 & Live coding solution & YouTube \\

         \hline

         12.50 - 14.00 & Lunch & - \\

         \hline

         14.00 - 14.25 & Services: exercise & Teams \\

         \hline

         14.25 - 14.50 & Live coding solution & YouTube \\

         \hline

         14.50 - 15.00 & Break & - \\

         \hline

         15.00 - 15.25 & Docker: exercise & Teams \\

         \hline

         15.25 - 15.50 & Live coding solution & YouTube \\

         \hline

         15.50 - 16.00 & Close & YouTube \\

         \hline

    \end{tabular}

\end{table}

\end{centering}

\end{frame}

%==============================

\begin{frame}[fragile]{Exercises}

There is an \href{https://github.com/martinteaching/sustainability/blob/master/workshops/kcl/2025/workshop-exercises.md}{\textbf{exercise}} available for each session, based on the current topic.

\pause

The first part of each session will give you time \textbf{individually} to review and make a start on the exercise using \href{https://github.com/martinteaching/sustainability/tree/master/workshops/kcl/2025#github-codespaces-recommended}{GitHub Codespaces}.

\begin{itemize}

    \item Also time to \textbf{catch up on the online material} for the current topic if needed.
    
    \item Focus on the \textbf{`in practice'} videos.

\end{itemize}

\pause 

The second will be a \textbf{collaborative} coding session, during which we will construct an answer to the exercise together.

\begin{itemize}

  \item I will share a \emph{Visual Studio Code} \textbf{Live Share} link (with `\textit{at-keyboard}' coding as a backup).

  \item We will take it in turns to be \textbf{in charge of the editor}, with those not currently coding \textbf{helping with the solution}.

\end{itemize}

\end{frame}

%==============================

\begin{frame}[fragile]{Assessment}

A requirement for the workshop.

Please \textbf{engage sufficiently with the collaborative coding sessions} in order to receive a `pass' mark.

\end{frame}

%==============================

\begin{frame}[fragile]{Technology Drift}

The material for this course was recorded a \textbf{little while ago}, which might seem strange for any technology-related workshop, given that technologies change all the time.

Because the topics we focus on are so \textbf{fundamental}, the course doesn't suffer too much from this \emph{technology drift}.
However, there are some \textbf{notable changes} from when the material was recorded:

\begin{itemize}
    \item \mintinline[bgcolor=bg]{bash}{docker-compose} to \mintinline[bgcolor=bg]{bash}{docker compose}
    \item I used to \emph{love} semicolons...
\end{itemize}

I am \textbf{continually reviewing} whether the material needs to be re-recorded.

\end{frame}

%==============================

\begin{frame}[fragile]{A quick word on how I like to run sessions...}

I am more of a \textbf{generalist} (`jack of all trades') than a \textbf{specialist}.

Therefore, likely people here who know \textbf{way more} about aspects of this technology than I do.

Correct me, challenge me and tell me better ways to do things -- I am very open to \textbf{learning more}.

Most importantly, collaborate and help \textbf{each other}.

\end{frame}

%==============================

\begin{frame}[fragile]{KCLHI and software sustainability: microservices}

A \textbf{microservice architecture} is really what we are building towards in this course, as they epitomise sustainable software.

An approach we use in our group, King's College London Health Informatics (KCLHI).

\begin{itemize}

    \item \href{https://kclhi.org}{kclhi.org}.
    
    \item \href{https://github.com/kclhi}{github.com/kclhi}.
    
\end{itemize}

Examples:

\begin{itemize}
    
    \item `\textbf{Phenoflow}' - \href{https://github.com/kclhi/phenoflow}{github.com/kclhi/phenoflow}. 
    \item `\textbf{REFLECT}' - \href{https://github.com/kclreflect}{github.com/kclreflect}.   
    \item `\textbf{CONSULT}' - \href{https://github.com/kclconsult}{github.com/kclconsult}.
    \item `\textbf{TMRWeb}' - \href{https://github.com/kclconsult/tmrweb}{github.com/kclconsult/tmrweb}.

\end{itemize}

\end{frame}

\end{document}

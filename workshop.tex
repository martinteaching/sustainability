\documentclass{article}

%Packages

\usepackage[utf8]{inputenc}

\usepackage[parfill]{parskip}

\usepackage[colorlinks=true,urlcolor=codeblue]{hyperref}

\usepackage{minted}

%Document

\title{`Software Sustainability: If a tree falls in a forest...' -
King's workshop prerequisites} \author{Martin Chapman} \date{Wednesday
9th December 2020}

\begin{document}

\definecolor{bg}{rgb}{0.95,0.95,0.95}
\definecolor{codeblue}{rgb}{0,0.3,0.6}

\maketitle

\section{Overview}

``If a tree falls in a forest, and no one is there to hear it, does it
make a sound?''. This thought experiment suggests that something cannot
exist, if it exists in isolation. The same is true of research software,
which can only truly make an impact if it continues to be used over
time. We refer to software that is able to stand the test of time as
sustainable. In the development of research software, there are several
steps we can take to ensure our software is sustainable, and is still used
in 5, 10 or even 20 years from now. This course will explore these steps,
which include the use of Git, service-based architectures, and Docker. In
producing sustainable software, not only do we ensure the longevity of
our work, but we also ensure that our research is as open as possible,
and, critically, our results are reproducible.

Attendees at the workshop will build a novel piece of software,
incrementally improving its sustainability, until they develop a product
capable of existing within the workbenches of the next generation of
researchers.

\section{Prerequisites}

\subsection{Video course}

Prior to attending the session, attendees should have reviewed the
recorded video course on software sustainability.

\textbf{YouTube playlist link}: \newline
\href{https://www.youtube.com/playlist?list=PLxyHJ\_wep1\_DPbvtFl\_-EGyoz2pVt-n1\_}{https://www.youtube.com/playlist?list=PLxyHJ\_wep1\_DPbvtFl\_-EGyoz2pVt-n1\_}

As the session will be entirely practical, this material will not be
presented in the session, and instead the attendees will complete tasks
based upon the recorded content. These tasks will be released shortly
before the session.

\subsection{Installation}

\subsubsection{Direct}

The following pieces of software should be installed prior to attending
the session, and preferably prior to engaging with the recorded
material. In order of importance:

\begin{enumerate}

    \item \textbf{Python 3}
    (\href{https://www.python.org/downloads/}{https://www.python.org/downloads/})

    Confirm Python 3 is installed by issuing the command
    \mintinline[bgcolor=bg]{bash}{python3 --version} on a terminal.

    \item \textbf{nodejs}
    (\href{https://nodejs.org/en/download/}{https://nodejs.org/en/download/})

    Confirm nodejs is installed by issuing the command
    \mintinline[bgcolor=bg]{bash}{node --version} on a terminal

    \item \textbf{Git}
    (\href{https://git-scm.com/book/en/v2/Getting-Started-Installing-Git}{https://git-scm.com/book/en/v2/Getting-Started-Installing-Git})

    Confirm git is installed by issuing the command
    \mintinline[bgcolor=bg]{bash}{git --version} on a terminal.

    \item \textbf{Docker}
    (\href{https://docs.docker.com/get-docker/}{https://docs.docker.com/get-docker/})

    Confirm docker is installed, by running the command
    \mintinline[bgcolor=bg]{bash}{docker run hello-world} on a terminal.

    \item \textbf{Docker Compose}
    (\href{https://docs.docker.com/compose/install/}{https://docs.docker.com/compose/install/})

    Confirm Docker Compose is installed by running the command \newline
    \mintinline[bgcolor=bg]{bash}{docker-compose --version}

\end{enumerate}

It is recommended that all these tools are installed directly to the
attendees machine. Unfortunately, support with installation cannot be
provided during the workshop.

\subsubsection{Virtual Machine}

If this installation process is
not possible, a virtual machine image is available
(\href{https://www.dropbox.com/s/p20rtne795yh3vb/sustainable.ova?dl=1}{https://www.dropbox.com/s/p20rtne795yh3vb/sustainable.ova?dl=1}),
within which all the required software has been pre-installed. This
image can be run using software such as VirtualBox
(\href{https://www.virtualbox.org/wiki/Downloads}{https://www.virtualbox.org/wiki/Downloads}).

\subsection{Slack}

Slack will be used for text communication during the
workshop. Attendees are also welcome to ask questions, such as
regarding software installation, prior to the session. Please
sign up for the `sustainability2020` workspace here, and
ensure you have access on the day of the workshop: \newline
\href{https://join.slack.com/t/sustainability2020hq/signup}{https://join.slack.com/t/sustainability2020hq/signup}

Any issues with accessing these resources should be reported to
\href{mailto:martin.chapman@kcl.ac.uk}{martin.chapman@kcl.ac.uk}.

\section{Workshop}

The workshop itself will be streamed via YouTube, with breakout rooms
to receive support available via Teams. Links to these platforms will
be made available to attendees before the session. The structure of the
day will be as follows:

\begin{table}[h!]

    \centering \begin{tabular}{c|c|c}
         
         Time & Activity & Platform  \\ 
         
         \hline 
         
         10.00 - 10.10 & Introduction & YouTube \\ 
         
         \hline 
         
         10.10 - 10.30 & Coding practice: exercise & Teams \\ 
         
         \hline 
         
         10.30 - 10.50 & Live coding solution & YouTube \\ 
         
         \hline 
         
         10.50 - 11.00 & Break & - \\ 
         
         \hline 
         
         11.00 - 11.25 & Version control: exercise & Teams \\ 
         
         \hline 
         
         11.25 - 11.50 & Live coding solution & YouTube \\ 
         
         \hline 
         
         11.50 - 12.00 & Break & - \\ 
         
         \hline 
         
         12.00 - 12.25 & Testing: exercise & Teams \\
         
         \hline 
         
         12.25 - 12.50 & Live coding solution & YouTube \\ 
         
         \hline
         
         12.50 - 14.00 & Lunch & - \\ 
         
         \hline 
         
         14.00 - 14.25 & Services: exercise & Teams \\ 
         
         \hline 
         
         14.25 - 14.50 & Live coding solution & YouTube \\ 
         
         \hline 
         
         14.50 - 15.00 & Break & - \\ 
         
         \hline 
         
         15.00 - 15.25 & Docker: exercise & Teams \\ 
         
         \hline 
         
         15.25 - 15.50 & Live coding solution & YouTube \\ 
         
         \hline 
         
         15.50 - 16.00 & Close & YouTube \\ 
         
         \hline
         
    \end{tabular}
\end{table}

The exercise to be completed on the day, which will include assessment
details, will also be made available to attendees prior to the session.

\end{document}

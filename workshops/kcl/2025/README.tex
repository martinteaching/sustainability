\documentclass{article}

%Packages

\usepackage[utf8]{inputenc}

\usepackage[parfill]{parskip}

\usepackage[colorlinks=true,urlcolor=codeblue]{hyperref}

\usepackage[]{minted}

\usepackage[normalem]{ulem}

%Document

\title{Software Sustainability - King's workshop prerequisites} 
\author{Martin Chapman} 
\date{Friday 2nd May 2022}

\begin{document}

\definecolor{bg}{rgb}{0.95,0.95,0.95}
\definecolor{codeblue}{rgb}{0,0.3,0.6}

\maketitle

\section{Overview}

Please read the \href{https://github.com/martinteaching/sustainability#overview}{course overview}.

Attendees at this workshop will build a novel piece of software, incrementally improving its sustainability, until they develop a product capable of existing within the workbenches of the next generation of researchers.

\section{Prerequisites}

\subsection{Video course}

Please watch the \href{https://github.com/martinteaching/sustainability#material}{linked course material}.

As the workshop will be entirely practical, this material will not be presented, and instead the attendees will develop their software by completing a set of exercises based upon the recorded content. 

\sout{The exercise to be completed on the day, which will include an assessment component, will also be made available to attendees prior to the workshop.} Exercises are available \href{https://github.com/martinteaching/sustainability/blob/master/workshops/kcl/2025/workshop-exercises.md}{here}.

\subsection{Installation}

\subsubsection{GitHub Codespaces (recommended)}

Attendees should familiarise themselves with how to use \href{https://github.com/codespaces}{GitHub Codespaces} prior to attending the workshop, and preferably prior to engaging with the recorded material.

Students should confirm that they can interact with the following software within a codespace.
In order of importance:

\begin{enumerate}

    \item \textbf{Python 3}
    (\href{https://www.python.org/downloads/}{https://www.python.org/downloads/})

    Confirm Python 3 is available by issuing the command
    \mintinline[bgcolor=bg]{bash}{python --version} on a terminal.

    \item \textbf{nodejs}
    (\href{https://nodejs.org/en/download/}{https://nodejs.org/en/download/})

    Confirm nodejs is available by issuing the command
    \mintinline[bgcolor=bg]{bash}{node --version} on a terminal.

    \item \textbf{Git}
    (\href{https://git-scm.com/book/en/v2/Getting-Started-Installing-Git}{https://git-scm.com/book/en/v2/Getting-Started-Installing-Git})

    Confirm git is available by issuing the command
    \mintinline[bgcolor=bg]{bash}{git --version} on a terminal.

    \item \textbf{Docker}
    (\href{https://docs.docker.com/get-docker/}{https://docs.docker.com/get-docker/})

    Confirm docker is available by issuing the command
    \mintinline[bgcolor=bg]{bash}{docker run hello-world} on a terminal.

    \item \textbf{Docker Compose}
    (\href{https://docs.docker.com/compose/install/}{https://docs.docker.com/compose/install/})

    Confirm Docker Compose is available by issuing the command \newline
    \mintinline[bgcolor=bg]{bash}{docker compose}

\end{enumerate}

Note that support with these Codespace prerequisites cannot be provided during the workshop.

\subsubsection{Direct (backup)}

In case of any issues with GitHub Codespaces, attendees are encouraged to install the software listed above directly to their machines using the links provided. 

\emph{Note: if you receive a permissions error when running the Docker command on a Linux machine, you may be required to run the command as the root user by prefixing it with the \mintinline[bgcolor=bg]{bash}{sudo} keyword. 
To permanently allow a non-root user access to Docker you can add a user (e.g. `user') to the docker Unix group with the command \mintinline[bgcolor=bg]{bash}{sudo usermod -aG docker user}.}

\section{Workshop}

Attendees may wish to review the \href{https://github.com/martinteaching/sustainability/blob/master/workshops/kcl/2025/workshop-slides.md}{workshop slides} in advance to understand the structure of the workshop itself.

\subsection{Visual Studio Code}

Visual Studio Code, which may be used for collaborative coding during the workshop, can be used through the browser, however students may also like to install this software locally.

\emph{Problems with any of the resources listed above should be reported to Martin Chapman (martin.chapman@kcl.ac.uk) prior to the workshop.}

\end{document}
